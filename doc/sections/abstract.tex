\begin{abstract}
In dieser Arbeit wurden verschiedene bekannte Verfahren zur Time-Scale Modification untersucht. Diese Verfahren wurden in JavaScript implementiert, auf einer Weboberfläche dargestellt und anschliessend ausgewertet. Es wurde gezeigt, warum diese Verfahren gerechtfertigt sind, wie sie funktionieren und welche Entscheidungen bei der Implementierung getroffen wurden. Abschliessend wurde analysiert, wie sich die verschiedenen Methoden auf das Audiosignal auswirken und wie gut sie die Tonhöhe und die Zeitstruktur bewahren. Da keine dieser Verfahren eine perfekte Time-Scale Modification ohne hörbare Artefakte ermöglicht, wurden diese Artefakte aufgezeigt und analysiert. Dadurch konnte ermittelt werden, welche Methode für welchen Anwendungsfall am besten geeignet ist, welche Artefakte sie erzeugt und durch welche Tricks sich diese optimieren lassen.
\end{abstract}  

\keywords{Audio, Signal Processing, JavaScript, Web}